
\rSec0[intro]{General}

\rSec1[intro.scope]{Scope}

\pnum
This Technical Specification describes extensions to the C++ 
Programming Language (\ref{intro.refs}) that
enables definition of coroutines. These extensions include 
new syntactic forms and modifications to existing language semantics.

\pnum
The International Standard, ISO/IEC 14882, provides important context
and specification for this Technical Specification. This document is 
written as a set of changes against that specification. Instructions
to modify or add paragraphs are written as explicit instructions. 
Modifications made directly to existing text from the International
Standard use \added{underlining} to represent added text and
\removed{strikethrough} to represent deleted text.

\rSec1[intro.ack]{Acknowledgements}

This work is the result of collaboration of researchers in industry and academia, including CppDes Microsoft group and the WG21 study group SG1. We wish to thank people who made valuable contributions within and outside these groups, including Jens Maurer, Artur Laksberg, Chandler Carruth, Gabriel Dos Reis, Deon Brewis, Jonathan Caves, James McNellis, Stephan T. Lavavej, Herb Sutter, Pablo Halpern, Robert Schumacher, Michael Wong, Niklas Gustafsson, Nick Maliwacki, Vladimir Petter, Shahms King, Slava Kuznetsov, Tongari J, Lawrence Crowl, and many others not named here who contributed to
the discussion.

\rSec1[intro.refs]{Normative references}

\pnum
The following referenced document is indispensable for the
application of this document. For dated references, only the
edition cited applies. For undated references, the latest edition
of the referenced document (including any amendments) applies.

\begin{itemize}
\item ISO/IEC 14882:2014, \doccite{Programming Languages -- \Cpp}
\end{itemize}

ISO/IEC 14882:2014 is hereafter called the \defn{\Cpp Standard}.
%
Beginning with section 1.4 below, all clause and section numbers, titles,
and symbolic references in [brackets] refer to the corresponding elements of the \Cpp Standard. Sections 1.1 through 1.5 of this Technical
Specification are introductory material and are unrelated to the similarly-numbered sections of the \Cpp Standard.

% NOTES: N4302 use transaction memory edits to fix this. 

%\rSec1[intro.defs]{Terms and definitions}
%
%Add the definitions of ``suspend-resume-point'' and ``coroutine''.
%
%\setcounter{subsection}{26}
%\begin{quote}
%\indexdefn{suspend-resume-point}%
%\definition{suspend-resume-point}{defns.suspend.resume}
%A point in a \grammarterm{function-body}
%where evaluation of a function can be suspended
%with possibility of resuming it later
%via a call to a member function of a
%\tcode{coroutine_handle} object associated with the suspended function.
%\end{quote}
%
%\begin{quote}
%	\indexdefn{coroutine}%
%	\definition{coroutine}{defns.coroutine.function}
%	A function defined with \grammarterm{function-body} that
%	contains one or more suspend-resume-points.
%\end{quote}

%%
%% Implementation compliance
%%
\rSec1[intro.compliance]{Implementation compliance}

\pnum
Conformance requirements for this specification are the same as those 
defined in section 1.4 of the \Cpp Standard. 
\enternote 
Conformance is defined
in terms of the behavior of programs.
\exitnote

%%
%% Feature-testing recommendations
%%
\rSec1[intro.features]{Feature-testing recommendations (Informative)}

An implementation that provides support for this Technical Specification shall define the feature test macro in Table~\ref{tab:info.features}.

\begin{floattable}{Feature-test macros for coroutines}{tab:info.features}
{lll}
\topline
\lhdr{Name} & \chdr{Value} & \rhdr{Header} \\
\capsep
\tcode{__cpp_coroutine}  & \tcode{201511} & \textit{predeclared}      \\
\end{floattable}

%\rSec1[intro.ack]{Acknowledgments}
%
%\pnum
%The design of this specification is based, in part, on a concept 
%specification of the algorithms part of the C++ standard library, known 
%as ``The Palo Alto'' report (WG21 N3351), which was developed by a large 
%group of experts as a test of the expressive power of the idea of 
%concepts. Despite syntactic differences between the notation of the 
%Palo Alto report and this Technical Specification, the report can be seen as a 
%large-scale test of the expressiveness of this Technical Specification.

\setcounter{section}{8}
\rSec1[intro.execution]{Program execution}
Modif





























y paragraph 7 to read:
\begin{quote}
\setcounter{Paras}{6}
\pnum 
An instance of each object with automatic storage 
duration~(\cxxref{basic.stc.auto}) is associated with each entry into its 
block. Such an object exists and retains its last-stored value during 
the execution of the block and while the block is suspended (by a call 
of a function\added{, suspension of a coroutine (\ref{dcl.fct.def.coroutine})}, 
or receipt of a signal). 
\end{quote}
