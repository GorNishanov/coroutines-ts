
%%
%% Declarators
%%
\rSec0[dcl.decl]{Declarators}

\setcounter{section}{3}
\rSec1[dcl.fct.def]{Function definitions}

Add subsection \ref{dcl.fct.def.resumable}

\setcounter{subsection}{3}
\rSec2[dcl.fct.def.resumable]{Resumable Functions}

\rSec3[resumable.definitions]{Definitions}

\pnum
A function is considered \term{resumable} if it contains
one or more \term{suspend-resume point}{}s. 
Suspend-resume points are created by \tcode{await} operator (\ref{expr.await}) in potentially-evaluated expression,
\tcode{yield} statement (\ref{stmt.yield}) 
or \tcode{for await} statement (\ref{stmt.for.await}).

%\pnum
%\term{Initial call} to a resumable function occ
%Resume call to a suspended resumable function
%Destroy call to a suspended resumable function
%resumable function instance
%\pnum
%Suspension of a resumable function 

\pnum
Execution of a resumable function is suspended when it reaches a suspend-resume point.
A suspended resumable function can be resumed
to continue execution by invoking
resumption member functions (\ref{resumable.handle.resumption}) of an object of \tcode{resumable_handle<P>} type
associated with this instance of a resumable function, where type P
is determined by insantiating struct template \tcode{std::experimental::resumable_traits} as described below. 

\pnum

\pnum
For a resumable function \tcode{f}, Let \tcode{R} be a return type and $P_1$, $P_2$, ..., $P_n$
where $n >= 0$ be types of parameters. If \tcode{f} is a non-static member function then $P_1$ denotes the type of implicit this parameter. 
Coroutine traits for function \tcode{f} is an insantiation of
struct template \tcode{std::experimental::resumable_traits<R,P1,...,PN>}.

\pnum
Coroutine traits determine what 


\pnum Given function \tcode{R f(T1 a, T2 b) S} where \tcode{S} is \grammarterm{function-body} with one or more suspend-resume points, 

\begin{codeblock}
	R f(T1 a, T2 b) {
		using _Traits = std::experimental::resumable_traits<R,T1,T2>;
		using _Alloc = _Traits::get_allocator(a,b);
		using _Promise = _Traits::promise_type;
		struct _Context {
			_Alloc __alloc;
			_Promise __promise;
			T1 a;
			T2 b;
			@\textit{unspecified-type _room_for_temps_and_locals}@;
			
			_Context(T1& a, T2& b) : a(move(a)), b(move(b)) {}
			
			void operator() {
				try { 
					@\textit{function-body}@ 
				} catch(...) {
					<handle exception as requested by the resumable promise>
				}
				__end:  
				if (__promise.final_suspend()) {
					<suspend-resume-point>
					std::terminate();
				}			
			}
		};
		_Alloc alloc
		if (!__pr.initial_suspend()) {
			__resume_f(context);
		}
	}
	
	void __resume_f(_Context* context) {
	<
}
\end{codeblock}

%\begin{codeblock}
%R f(T1 a, T2 b) {
%  // specialize resumable_traits to discover customization points
%  using _Traits = std::experimental::resumable_traits<R,T1,T2>;
%  <allocate memory for automatic variables>
%  <transfer parameters as needed>
%  <prepare return value>
%  if (!__pr.initial_suspend()) {
%    __resume_f(context);
%  }
%}
%
%void __resume_f(_Context* context) {
%  try {
%  } catch(...) {
%  <handle exception as requiested by the resumable promise>
%}
%__end:  
%if (__promise.final_suspend()) {
%	<suspend-resume-point>
%	std::terminate();
%}
%<
%}
%\end{codeblock}
