
\setcounter{chapter}{17}
\rSec0[language.support]{Language support library}

\rSec1[support.general]{General}

Add a row to Table~\ref{tab:lang.sup.lib.summary} for \tcode{<experimental/resumable>}


\begin{libsumtab}{Language support library summary}{tab:lang.sup.lib.summary}
	\cxxref{support.types}       & Types                     &   \tcode{<cstddef>}   \\ \rowsep
	&                           &   \tcode{<limits>}    \\
	\cxxref{support.limits}      & Implementation properties &   \tcode{<climits>}   \\
	&                           &   \tcode{<cfloat>}    \\ \rowsep
	\cxxref{cstdint}             & Integer types             & \tcode{<cstdint>}     \\ \rowsep
	\cxxref{support.start.term}  & Start and termination     &   \tcode{<cstdlib>}   \\ \rowsep
	\cxxref{support.dynamic}     & Dynamic memory management &   \tcode{<new>}       \\ \rowsep
	\cxxref{support.rtti}        & Type identification       &   \tcode{<typeinfo>}  \\ \rowsep
	\cxxref{support.exception}   & Exception handling        &   \tcode{<exception>} \\ \rowsep
	\cxxref{support.initlist}    & Initializer lists & \tcode{<initializer_list>}    \\ \rowsep
	\added{\ref{support.resumable}} 
  & \added{Resumable functions support} 
  & \added{\tcode{<experimental/resumable>}}    \\ \rowsep
	&                           &   \tcode{<csignal>}   \\
	&                           &   \tcode{<csetjmp>}   \\
	&                           &   \tcode{<cstdalign>} \\
	\cxxref{support.runtime}     & Other runtime support     &   \tcode{<cstdarg>}   \\
	&                           &   \tcode{<cstdbool>}  \\
	&                           &   \tcode{<cstdlib>}   \\
	&                           &   \tcode{<ctime>}     \\
\end{libsumtab}


Add section \ref{support.resumable}

\setcounter{section}{10}
\rSec1[support.resumable]{Resumable functions support library}

\pnum
The header
\tcode{<experimental/resumable]>}
defines several types supporting resumable functions in a \Cpp program.

\synopsis{Header \tcode{<experimental/resumable>} synopsis}

\indextext{\idxhdr{experimental/resumable}}%
\indexlibrary{\idxhdr{experimental/resumable}}%
\begin{codeblock}
namespace std {
namespace experimental {
  template <typename R, typename... ArgTypes>
    class coroutine_traits;
	
  template <typename Promise = void>
    class coroutine_handle;		

  template <> class coroutine_handle<void>;
	
  bool operator == (coroutine_handle<> x, coroutine_handle<> y) noexcept;
  bool operator < (coroutine_handle<> x, coroutine_handle<> y) noexcept;			
  bool operator != (coroutine_handle<> x, coroutine_handle<> y) noexcept;
  bool operator <= (coroutine_handle<> x, coroutine_handle<> y) noexcept;			
  bool operator >= (coroutine_handle<> x, coroutine_handle<> y) noexcept;
  bool operator > (coroutine_handle<> x, coroutine_handle<> y) noexcept;			
}
}
\end{codeblock}

\rSec2[resumable.traits]{Class template \tcode{coroutine_traits}}

\indexlibrary{\idxcode{coroutine_traits}}%
\begin{codeblock}
namespace std {
namespace experimental {
  template <typename R, typename... Ts>
  class coroutine_traits {
  public:
    template <typename... Us>
      static auto get_allocator(Us&&...); // optional
			
    static auto get_return_object_on_allocation_failure() noexcept; // optional
			
    using promise_type = typename R::promise_type;
  };
} // namespace experimental
} // namespace std
\end{codeblock}

\pnum
The class
\tcode{coroutine_traits}
provides bla-bla as specified in~\ref{dcl.fct.def.resumable}.
class for the types of objects thrown as exceptions by
\Cpp standard library components, and certain
expressions, to report errors detected during program execution.

\pnum
The \tcode{coroutine_traits} may be specialized for user-defined types 
to indicate that such types are eligible


\rSec2[resumable.handle]{Class \tcode{coroutine_handle}}

\indexlibrary{\idxcode{coroutine_handle}}%
\begin{codeblock}
namespace std {
namespace experimental {
  template <typename R, typename... Ts>
  class coroutine_traits {
  public:
    template <typename... Us>
    auto get_allocator(Us&&...);
				
    using promise_type = typename R::promise_type;
  };
}
}
\end{codeblock}

\pnum
The class
\tcode{coroutine_traits}
defines the base
class for the types of objects thrown as exceptions by
\Cpp standard library components, and certain
expressions, to report errors detected during program execution.
