
\rSec0[stmt.stmt]{Statements}%

\setcounter{section}{4}
\rSec1[stmt.iter]{Iteration statements}%
Add underlined text to paragraph 1.

\begin{quote}
\pnum
Iteration statements specify looping.

\indextext{statement!\idxcode{while}}%
\indextext{statement!\idxcode{do}}%
\indextext{statement!\idxcode{for}}%
%
\begin{bnf}
	\nontermdef{iteration-statement}\br
	\terminal{while (} condition \terminal{)} statement\br
	\terminal{do} statement \terminal{while (} expression \terminal{) ;}\br
	\terminal{for (} for-init-statement condition\opt \terminal{;} expression\opt \terminal{)} statement\br
	\terminal{for (} for-range-declaration \terminal{:} for-range-initializer \terminal{)} statement\br
	\added{\terminal{for await (} for-range-declaration \terminal{:} for-range-initializer \terminal{)} statement}
\end{bnf}
\end{quote}

Add subclause 6.5.5.

\setcounter{subsection}{4}
\rSec2[stmt.for.await]{The await-for statement}%

\pnum
A \tcode{for await} statement of the form

\begin{ncbnf}
	\terminal{for await(} for-range-declaration : expression \terminal{)} statement
\end{ncbnf}

is equivalent to

\begin{codeblock}
	{
		auto && __range = range-init;
		for ( auto __begin = await begin-expr,
		__end = end-expr;
		__begin != __end;
		await ++__begin ) {
			@\textit{for-range-declaration}@ = *__begin;
			@\textit{statement}@
		}
	}
\end{codeblock}

where \tcode{__range}, \tcode{__begin}, 
\textit{range-init}, \textit{begin-expr} and \textit{end-expr} are defined as in the range-based for statement \cxxref{stmt.ranged}.

\setcounter{section}{5}
\rSec1[stmt.jump]{Jump statements}%
\setcounter{subsection}{2}
\rSec2[stmt.return]{The \tcode{return} statement}%
\indextext{\idxcode{return}}%
\indextext{function~return|see{\tcode{return}}}%

Modify paragraphs 1 through 3 as follows.

\begin{quote}
\pnum
A function returns to its caller by the \tcode{return} statement.
\added{A resumable function also returns to its caller 
when suspended at suspend-resume point.}

\pnum
\added{In a non-resumable function a}\removed{A} return statement
with neither an \grammarterm{expression} nor a \grammarterm{braced-init-list}
can be used only in functions
that do not return a value, that is, a function with the return type
\cv\ \tcode{void}, a constructor~(\cxxref{class.ctor}), or a
destructor~(\cxxref{class.dtor}).
\indextext{\idxcode{return}!constructor~and}%
\indextext{\idxcode{return}!constructor~and}%
A return statement with an expression of non-void type can be used only
in functions returning a value; the value of the expression is returned
to the caller of the function.
\indextext{conversion!return~type}%
The value of the expression is implicitly converted to the return type of the
function in which it appears. A return statement can involve the
construction and copy or move of a temporary object~(\cxxref{class.temporary}).
\enternote
A copy or move operation associated with a return statement may be elided or
considered as an rvalue for the purpose of overload resolution in
selecting a constructor~(\cxxref{class.copy}).
\exitnote A return statement with a \grammarterm{braced-init-list} initializes the object or reference to be returned from the function by copy-list-initialization~(\cxxref{dcl.init.list}) from the specified initializer list. \enterexample

\begin{codeblock}
	std::pair<std::string,int> f(const char* p, int x) {
		return {p,x};
	}
\end{codeblock}
\exitexample

Flowing off the end of a function is equivalent to a \tcode{return} with
no value; this results in undefined behavior in a value-returning
function.

\pnum
A return statement with an expression of type \tcode{void}
can be used only in \added{non-resumable} functions with a return type of
\cvqual{cv} \tcode{void}
\added{or resumable functions with eventual return type of \cvqual{cv} \tcode{void}}; 
the expression is evaluated just before the function
returns to its caller.
\end{quote}

%Add underlined text to paragraph 1:
%\pnum
%A function returns to its caller by the \tcode{return} statement
%\added{or by reaching suspend-resume-point}.

Add paragraph 4.

\begin{quote}
\setcounter{Paras}{3}
\pnum 
In a resumable function return statement is replaced with
a call to __pr.set_result, where _p
\end{quote}

\setcounter{section}{8}
\rSec1[stmt.yield]{Yield statement}%

Add this section after \cxxref{stmt.ambig}.

Hello there


