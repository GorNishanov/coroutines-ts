
\rSec0[stmt.stmt]{Statements}%

\setcounter{section}{4}
\rSec1[stmt.iter]{Iteration statements}%
Add underlined text to paragraph 1.

\begin{quote}
\pnum
Iteration statements specify looping.

\indextext{statement!\idxcode{while}}%
\indextext{statement!\idxcode{do}}%
\indextext{statement!\idxcode{for}}%
%
\begin{bnf}
	\nontermdef{iteration-statement}\br
	\terminal{while (} condition \terminal{)} statement\br
	\terminal{do} statement \terminal{while (} expression \terminal{) ;}\br
	\terminal{for (} for-init-statement condition\opt \terminal{;} expression\opt \terminal{)} statement\br
	\terminal{for (} for-range-declaration \terminal{:} for-range-initializer \terminal{)} statement\br
	\added{\terminal{for await (} for-range-declaration \terminal{:} for-range-initializer \terminal{)} statement}
\end{bnf}
\end{quote}

Add subclause 6.5.5.

\setcounter{subsection}{4}
\rSec2[stmt.for.await]{The await-for statement}%

\pnum
A \tcode{for await} statement of the form

\begin{ncbnf}
	\terminal{for await(} for-range-declaration : expression \terminal{)} statement
\end{ncbnf}

is equivalent to

\begin{codeblock}
	{
		auto && __range = range-init;
		for ( auto __begin = await begin-expr,
		__end = end-expr;
		__begin != __end;
		await ++__begin ) {
			@\textit{for-range-declaration}@ = *__begin;
			@\textit{statement}@
		}
	}
\end{codeblock}

where \tcode{__range}, \tcode{__begin}, 
\textit{range-init}, \textit{begin-expr} and \textit{end-expr} are defined as in the range-based for statement \cxxref{stmt.ranged}.

\setcounter{section}{5}
\rSec1[stmt.jump]{Jump statements}%
\setcounter{subsection}{2}
\rSec2[stmt.return]{The \tcode{return} statement}%

%Add underlined text to paragraph 1:
%\pnum
%A function returns to its caller by the \tcode{return} statement
%\added{or by reaching suspend-resume-point}.

Add subclause 4

\setcounter{Paras}{3}
\pnum 
Within a resumable function bla bla

\setcounter{section}{8}
\rSec1[stmt.yield]{Yield statement}%

Add this section after \cxxref{stmt.ambig}.

Hello there


